\documentclass[11pt]{ctexart}
\usepackage[a4paper,width=150mm,top=25mm,bottom=25mm]{geometry}

\setmainfont{Caladea}

%% 也可以选用其它字库:
% \setCJKmainfont[%
%   ItalicFont=AR PL KaitiM GB,
%   BoldFont=Noto Sans CJK SC,
% ]{Noto Serif CJK SC}
% \setCJKsansfont{Noto Sans CJK SC}
% \renewcommand{\kaishu}{\CJKfontspec{AR PL KaitiM GB}}


\usepackage{minted}

\usepackage[breaklinks]{hyperref}
\usepackage{amsmath}
\newtheorem{theorem}{Theorem}
\newtheorem{lemma}{Lemma}
\newtheorem{proof}{Proof}[section]

\setlength{\parindent}{0pt}

\title{\Huge 第六讲  \ \ 整数规划}

\author{主讲教师:\  张国川
\\ 讲义整理: \ 温子奇,陈程,孔令祺}

\begin{document}

\maketitle

%概述本节主要内容、结构(类似摘要)


\section{整数规划实例}

\subsection{最大权匹配}
对于一个有权图,求一组边集,使其中每个顶点最多是集合中一条边的顶点,并且使得该集合的边权之和最大。\\

以下对问题进行建模:\\
\\
通过$x_e$表示一个边是否被选中:\\

$$
x=\left\{\begin{array}{ll}
1, & \text { when \quad } e \text { \quad is selected } \\
0, & \text { else } 
\end{array}\right.
$$

则最大权匹配问题可以描述如下:

$$
\begin{array}{c}
\max \sum_{e \in E} s_{e}x_{e} \\
\text { s.t. } \sum_{i \in e} x_{e} \leq 1 \\
x_{e} \in\{0,1\}
\end{array}
$$

其中第一个等式的$s_{e}$表示每个边的权值 \\

第二个等式的含义是每个端点只能属于一个边 \\


\subsection{0-1背包问题}

考虑$n$个物品,第$i$个物品大小为$s_i$,权重为$w_i$,每个背包大小为$C$,物品不能被拆分,要求在背包中装下尽可能多的物品。\\

与最大权问题相似,通过$x_e$表示是否选中某个物体\\

则问题可以描述如下:

$$
\begin{array}{l}
\max \sum_{i=1}^{n} w_{i} x_{i} \\
\text { s.t. } \sum_{i=1}^{n} s_{i} x_{i} \leq K \\
x_{i} \in\{0,1\}
\end{array}
$$

\subsection{0-1装箱问题}

记箱子大小为 $C$, 有 $n$ 个物品,第 $i$ 个物体大小为 $ai\ leq C$,至多用 $n$ 个箱子 , $y_i=1$ 表示使用该箱子,$y_i=0$ 则不使用。$x_{ij}=1$ 表示第 $i$ 个物体放入箱子$j$。则装箱问题的整数规划可以如下描述:
$$
\begin{array}{c}
\min \sum_{i=1}^{n} y_{i} \\
\text { s.t. } \sum_{i=1}^{n} x_{i j} a_{i} \leq C \cdot y_{i} \\
\sum_{i=1}^{n} x_{i j} \geq 1 \\
x_{i j} \in\{0,1\}, y_{i} \in\{0,1\}
\end{array}
$$

\subsection{TSP问题}

对于带权有向图$G=(V,E)$,顶点集为$V=(v_0,v_1,...,v_n)$,从任意一点$v_i$出发,途进所有点并返回$v_i$,并使路径的权重和最小。\\

利用$x_{ij}$表示是否选中从i到j的边,$w_{ij}$表示从i到j的边的权重,则问题先表述如下:
$$
\begin{array}{l}
\min \sum_{(i, j) \in E} w_{i j} x_{i j} \\
\text { s.t. } \sum_{j=1}^{n} x_{i j}=1, \forall j \in V \\
\quad \sum_{i=1}^{n} x_{i j}=1, \forall i \in V \\
x_{ij} \in \{0,1\}
\end{array}
$$

其中,第一个约束表示选中的所有端点的入度为1,第二个约束表示为所有端点的出度为1,两个约束合在一起表示每个顶点都只能经过一次。\\

但是注意到一个问题,如果存在多个不交叉的环路遍历了全部的顶点,约束也是符合的,但这显然不符合问题的定义,因此需要引入新的约束,使得多个环路的情况能够被排除,但是哈密尔顿回路满足约束。\\

为此,对于每一个顶点$v_i$引入辅助变量$u_i$,并且添加如下约束:
$$
u_{i}-u_{j}+nx_{i j} \leq n-1,1 \leq i, j \leq n, i \neq j
$$
可以注意到,这一约束不包含$i=0$的顶点,构建这一约束的目的在于:对于多环路的情况来说,必然存在环路不包含$i=0$的顶点,因此构建一个约束使得不包含$i=0$的顶点的环路不满足该约束,从而排除多环路的情况。由于哈密尔顿回路必然是一个环路,因此必然可以包含$i=0$的顶点,从而满足这一约束。\\

接下来说明这一约束是如何发挥作用的:\\

a. 对于多环路的情况,存在环路不包含$i=0$的顶点,则对于这一环路$C={v_{a0},v_{a1},v_{a2}...v_{an}}$,将需要满足如下约束:

$$
\begin{array}{l}
u_{a_0}-u_{a_1}+nx_{a_0a_1}\leq n-1 \\
u_{a_1}-u_{a_2}+nx_{a_1a_2}\leq n-1\\
...\\
u_{a_{n-1}}-u_{an}+nx_{a_{n-1}a_n}\leq n-1
\end{array}
$$
将上述约束相加,则对于辅助变量$u$将全部消去,从而变为:
$$
n\leq n-1
$$
显然不满足,因此多环路的情况将被排除。

b. 对于哈密尔顿回路,则必然包含顶点$v_0$。对哈密尔顿回路$H={v_0,v_1,v_2,...,v_n}$,令$u$按照如下规则取值:

$$
\begin{array}{1}
u_i=i
\end{array}
$$
带入新增约束中,表达式为:
$$
n-1 \leq n-1
$$
显然满足,哈密尔顿回路可以通过新增约束。


\subsection{最小(权)覆盖问题}
对于一个有权图(每个顶点具有一个权重),求一组顶点集,使每个边均有点在这个集合中,并使得这个集合的权重之和最小。\\

和上述建模方式同理,该问题可以描述为如下的整数规划问题:\\

$$
\begin{array}{lll}
\min & \sum_{i=1}^{n} w_{i} x_{i} & \\
\text { s.t. } & x_{i}+x_{j} \geq 1 & \forall(i, j) \in E \\
& x_{i} \in\{0,1\} & \forall i \in[\mathrm{n}]
\end{array}
$$\\


\section{整数规划解法}

\subsection{线性规划松弛}

在整数规划中,可以将$x_i \in \{0,1\}$的整数约束条件,松弛为$x_i \in [0,1]$的约束,这一过程可以理解为增加了松弛变量来完成。\\

并且在整数规划是求$min$的情况下,$x  \leq 1$的条件也是不必要的:\\
考虑1.5的最小覆盖问题的表述:

$$
\begin{array}{c}
\min \sum_{i=1}^{n} x_{i} \\
\text { s.t. } x_{i}+x_{j} \geq 1, \forall(i, j) \in E \\
x_{i j}>0
x_i \in [0,1]
\end{array}
$$
显然,在达到最优解时,$x_i \leq 1$,因此线性规划松弛后的最小覆盖问题可以表示为
:
$$
\begin{array}{c}
\min \sum_{i=1}^{n} x_{i} \\
\text { s.t. } x_{i}+x_{j} \geq 1, \forall(i, j) \in E \\
0 \leq x_i
\end{array}
$$

通过这一方式,可以将整数规划转化为线性规划来求解。

\subsection{总是整数解的情况}

对于部分整数规划问题,其线性规划松弛后得到的解总是整数解,例如最短路问题、最大流问题。原因在于其线性规划的约束矩阵$A$满足某一特殊性质,因此引入全幺模矩阵来说明这一点:

\textbf{幺模矩阵:} 方阵的行列式的值为$\pm 1$ \\
\textbf{全幺模矩阵:} 矩阵的所有非奇异子方阵都是幺模矩阵。

\begin{theorem}
当线性规划的约束矩阵$A$是全幺模矩阵时,线性规划的最优解是整数解。
\end{theorem}

\begin{proof}
首先说明行交换不会改变全幺模矩阵的性质:\\
任意子方阵的行列式都可以看成一个元素乘以该元素的余子式,而行(列)交换只会改变一行(列),且全幺模矩阵中元素只能为 0,1,-1,因此交换后的子方针的行列式依然为 0,1,-1。\\

注意到单纯形法对线性规划的最优解是$x_B=A^{-1}_{B}b$,整数规划中$b$均为整数,考虑$A^{-1}_{B}$,由于行交换不会改变全幺模矩阵的性质,故$\vert A_B\vert  = 1$,可得$A_B^{-1} = A_B^*/\vert A_B\vert$,则$A_B^{-1}$的元素也均为整数,证毕。
\end{proof}

以下介绍两个全幺模矩阵的实例:\\
\textbf{有向图的关联矩阵},
\textbf{二分图的关联矩阵};\\

可以通过数学归纳法证明其是全幺模矩阵,以下以有向图的关联矩阵为例证明:\\

\textbf{初始步骤}:对于 $n=1$ 的有向图,是全幺模矩阵。\\

\textbf{归纳假设}:设对于任意有向图 $n \leq k$ ,都是全幺模矩阵。\\

\textbf{归纳步骤}:对于任意有向图 $n=k+1$ 的子方阵,根据关联矩阵的定义,每一列至多存在两个非 0 元素,且若存在,至多存在一个 1,一个 -1。\\

1. 若该子方阵存在一列没有非 0 元素,那么该子方阵的行列式取值为 0;\\

2. 若该子方阵存在一列只有一个非 0 元素,由于该元素为 1 或 -1,该子方阵行列式的绝对值等于该元素余子式的绝对值。将原有向图去掉该元素对应的点和边后,这个余子阵可以看作是新的n-1个顶点有向图的子方阵,根据归纳假设,余子式的行列式为 0,1 或 -1,因此该子方阵的行列式取值为 0,1 或 -1;\\

3. 若该子方阵的每一列都有两个非 0 元素,对行进行累加得到零向量,即行向量线性相关,行列式为0;\\

故对于 n=k+1 个顶点的有向图的任意子方阵,其行列式的取值仍为 0,1 或 -1。\\

综上,有向图的关联矩阵是全幺模矩阵。\\


可以证明,具有如下性质的矩阵是全幺模矩阵:\\

设整数矩阵A元素为$0$或$\pm 1$,如果A的每列非零元素至多有两个,而且A的行可以分为两个子集I和J使得:\\
如果一列中有两个非零元素符号相同,则它们所在行分别属于I和K;\\
如果一列中有两个非零元素符号不同,则他们所在行同时属于I或J;\\


\subsection{一般解法}
\subsubsection{割平面法}

割平面法的思路在于通过构造新的约束条件,在最优解附近割掉一部分空间从而使得割掉最优非整数解的同时,保留最优整数解。\\

算法过程如下:\\

首先利用单纯形法求解线性规划松弛后的整数规划问题,得到典则形式:
$$
x_{i}+\sum_{k=m+1}^{n} a_{i k} x_{k}=b_{i}
$$
其中当$b_{i}$为整数时,最优解为整数,当$b_{i}$不为整数时:\\
1. 由于非基变量$x_{k} \geq 0$可得:$x_{i}+\sum_{k=m+1}^{n}\left\lfloor\bar{a}_{i k}\right\rfloor x_{k} \leq \bar{b}_{i k}$  \\
2. 由于整数规划的解为整数,则等式右侧也为整数,因此可得:$
x_{i}+\sum_{k=m+1}^{n}\left\lfloor\bar{a}_{i k}\right\rfloor x_{k} \leq\left\lfloor\bar{b}_{i k}\right\rfloor$
该不等式减去典则形式可得新的约束:
$$
\sum_{k=m+1}^{n}\left(\left\lfloor\bar{a}_{i k}\right\rfloor-\bar{a}_{i k}\right) x_{k} \leq\left\lfloor\bar{b}_{i k}\right\rfloor-\bar{b}_{i k}
$$
将该约束加入到之前的约束条件中,在原先的单纯形表中,用对偶单纯形法继续计算。迭代上述步骤,可以证明在有限步内得到整数最优解。


\subsubsection{分支定界法}

对线性规划松弛后的整数规划:
$$
\begin{aligned}
\min  \quad & c^{T} x \\
\text { s.t. } & A x \geq b \\
& x \geq 0
\end{aligned}
$$
若在线性规划最优解中$x^*_i=\bar{b_i}$不是整数,则在整数最优解中,$x_i$必满足以下两个条件中的一个:
$$
\begin{array}{c}
x_{i} \geq\left\lfloor\bar{b}_{i}\right\rfloor+1 \\
x_{i} \leq\left\lfloor\bar{b}_{i}\right\rfloor
\end{array}
$$

分别将两种约束加入到原有约束中,分别求解,经过有限步的分支后可以获得最优解。\\

不妨考虑最大化问题(当问题是最小化问题时需要反过来考虑),在分支定界法中,子规划所得到的新的最优解一定会低于父规划,对于每一个最优解的可以接受的范围,有如下定义:\\
\textbf{上界}:当前分支中最优的分数解的目标函数值\\
\textbf{下界}:当前最好的整数解的目标函数值,如果某一分支的最优解的值小于该值,则意味这该节点继续进行分支只会得到更差的点,则可以剪去。\\

根据上述定义,可以有如下的剪枝方式来减少计算:\\

\textbf{枯枝}:线性规划的最优解小于下界,则不可能得到更好的值,应该剪枝。\\
\textbf{明枝}:线性规划的最优解是整数解的节点,应该停止分支(由于继续分支得到的解必然差于该解)\\
\textbf{死枝}:线性规划没有可行解的节点,应当中止分支。\\
\textbf{活枝}:上述之外的都为活枝,可以继续进行分支。\\

当所有分支都结束时,下界将作为最优整数解。

\end{document}


