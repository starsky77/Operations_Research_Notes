\documentclass[11pt]{ctexart}
\usepackage[a4paper,width=150mm,top=25mm,bottom=25mm]{geometry}

\setmainfont{Caladea}

%% 也可以选用其它字库:
% \setCJKmainfont[%
%   ItalicFont=AR PL KaitiM GB,
%   BoldFont=Noto Sans CJK SC,
% ]{Noto Serif CJK SC}
% \setCJKsansfont{Noto Sans CJK SC}
% \renewcommand{\kaishu}{\CJKfontspec{AR PL KaitiM GB}}


\usepackage{minted}

\usepackage[breaklinks]{hyperref}
\usepackage{amsmath}
\newtheorem{theorem}{Theorem}
\newtheorem{lemma}{Lemma}
\newtheorem{proof}{Proof}[section]

\setlength{\parindent}{0pt}

\title{\Huge 第十二讲  \ \ 在线算法}

\author{主讲教师:\  XXX
\\ 讲义整理: \ XXX}

\begin{document}

\maketitle

%概述本节主要内容、结构(类似摘要)



\section{Competitive Analysis}

在线算法的性能通过一种类似于近似比的概念来衡量,即\textbf{competitive ratio};\\

若A表示在线算法,OPT表示相同问题的最优离线(offline)算法,\\
对于输入I,A(I)表示该算法的消耗,OPT(I)表示最优解的消耗,则competitive ratio定义如下:\\
设$c$为competitive ratio,对于最小化问题,$\forall I$,存在常数$\beta$使得\\
$$A(I) \leq c*OPT(I)+\beta$$
同时,我们称这一算法为c-competitive算法。

\section{ k-Server Problem}

\subsection{特殊情况:paging problem}
首先考虑一个操作系统或体系结构中的常见问题,它是k-Server Problem的一种特殊情况;\\

有容量为$k$ pages的cache以及容量为$n$ pages的主存;\\

访问cache中的数据没有消耗,而如果访问的数据不在cache中,则必须将其从主存中移动到cache中,记消耗为1。\\

目标:对于一段请求序列,要求总cost最小化。\\

\subsection{问题描述}
对于一个加权网络$G=(V,E)$,有$K$个server在初始分布在某些顶点上;\\
请求集合$\sigma=(r_1,r_2,……,r_n)$逐次到来(每个请求发生在一个顶点上)\\
当一个请求发生时,必须有一个server存在于这个顶点上,否则就需要分配一个servver前往这个顶点;\\
目标:将总的路程消耗(每个server移动路程的总和)最小化。\\

可以注意到,当网络为完全图时,且所有边权均为1时,k-Server Problem即:paging problem。

\subsection{贪心算法}

k-Server Problem的贪心算法:当每次请求发生时,派遣离这一顶点最近的server前往该顶点。\\

这一贪心算法并不能提供一个可观的解,考虑如下最坏情况:\\

对于一个只有三个顶点的图,同时有两个servers(server可以位于任意位置),顶点C位于顶点A,B之间,且$d(A,C)<d(B,C)$\\

发生的请求序列为:$\sigma=(A,B,C,A,C,A,C,A,C……)$\\

可以注意到,这种情况下会存在一个servers不断往返于A,C之间,而位于B的server并不会移动,而最优解显然是两个servers位于A与C,则在之后不会发生任何移动。



\subsection{k-Server on a Line}
接下来我们将给出一种算法,从较为特殊的情况开始,逐渐一般化,首先考虑一条线上的k-Server问题。

\subsubsection{Algorithm DC}
如果请求发生在server的凸包之外,则派遣距离它最近的server前往(和贪心一致);\\

如果请求发生在两个server之间,则两个servers将同时前往这一顶点(以相同的速度),先到达一方将进行服务,而此时未到达的一方停止;\\

注意这里“停止”的含义,实际上未到达的server仍然位于先前的顶点,但是如果之后的请求发生在相同位置,它将继续先前走过的路开始前进,这可以等价地视作server所在的这一点与请求发生的顶点之间的边的权重降低了。\\

继续考虑上述贪心算法中的最坏情况,这一算法可以有效避免这个情况。

\subsubsection{Analysis}
\begin{theorem}\\
DC算法是K-competitive算法;
\end{theorem}

\begin{proof}\\
通过Amortized Analysis的Potential Function方式来证明这一点。\\
我们将构造一个势能函数$\Phi$,它具有如下特点:\\

假设最优解OPT与DC算法,对于相同问题同时进行运算,\\
如果OPT将server移动了距离$d$,$\Phi$至多增加$kd$,\\
如果DC将server移动了距离$d$,$\Phi$最少减少$d$;\\
假设$\Phi_i$是第i次请求发生后势能函数的值;\\
则n次请求发生后势能函数的增量:$\Phi_n-\Phi_0$,最多为$k*OPT(\sigma)-DC(\sigma)$。\\

Potential Function如下:\\
$M_{min}$表示OPT的servers与DC的servers之间最小权完美匹配的cost。\\
$s_i$表示DC的第i个server所在的顶点\\
$\Sigma=\Sigma_{i<j}d(s_i,s_j)$\\
则势能函数为:$\Phi=k*M_{min}+\Sigma$\\

(PS:别问是怎么构造的,问就是试出来的)\\

可以依次考虑,一次请求下仅有OPT移动了server/仅有DC移动了server/双方都移动了server的情况,可以发现构造的函数是符合之前的要求的。\\
\end{proof}

\subsection{k-Server on a Tree}
在树图的情况下,考虑如下定义:\\

一个请求的\textbf{neighbor}:当一个server所在的顶点与请求所在的顶点之间的通路上没有其他server,则称这个server为该请求的neighbor。\\

当一个请求发生时,一个请求所有的neighbor都以相同的速度向请求派遣(执行方式与之前的k-Server on a Line一致)。\\

同样可以证明这也是一个K-competitive算法。

\subsection{k-Server on a General Network}

定义一个算法的configuration $C$ 为K个servers所在顶点的集合。

\subsubsection{Work-Function Algorithm (WFA)}
定义Working Function $W(C)$表示从初始configuration $C_0$ 经过请求序列$\sigma_i$,到最终的configuration $C$的最优cost。\\

函数$arg min()$表示使得()内函数式达到最小时变量的取值;\\

则对于请求$r_{i+1}$,该算法将移动server $s \in C$至点$r$\\
$s=arg min_{x \in C}{W(C-x+r)+d(x,r)}$\\

可以证明这是一个(2k-1)-competitive算法\\


\subsection{k-Server的相关猜想}
1. k-Server问题存在一个k-competitive的在线算法;\\

2. WFA本身是一个 k-competitve的算法(只不过没人给出约束为紧的例子)。\\

\end{document}


